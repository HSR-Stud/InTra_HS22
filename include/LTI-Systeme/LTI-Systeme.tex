% Needs Package: \usepackage{bm}
\section{LTI-Systeme}

$ y(t) = \mathcal{T}[x(t)]$

\begin{multicols}{2}
    \subsection*{Linearität und Zeitinvarianz}
    \begin{itemize}
        \item $\mathcal{T}[x_1(t) + x_2(t)] = y_1(t) + y_2(t)$
        \item $\mathcal{T}[k_a \cdot x(t)] = k_a \cdot y(t)$
        \item $\mathcal{T}[x(t-t_0) = y(t-t_0)]$
    \end{itemize}

    \subsection*{Kausalität}
    \begin{itemize}
        \item Jede Wirkung hat eine Ursache
        \item der Aktuelle y(t) Wert ist vom zukünftigen x(t) unabhängig
        \item Mathematische Systeme müssen nicht Kausal sein
        \item Physikalisch ist Kausalität ein Grundprinzip nach Newton
    \end{itemize}

    \subsection{Beschreibung von LTI Systemen}

    \subsubsection{Impulsantwort}
    Impulsfunktion $\delta(t)$ wird am Eingang des Systems angelegt,
    die Reaktion darauf am Ausgang nennt man die \textbf{Impulsantwort} \bm{$h(t)$}.
    Sie beschreibt ein LTI-System vollständig.

    $$ y(t) = \mathcal{T}[x(t)]
        = \int \limits _{-\infty} ^{infty} x(\tau) \cdot h(t-\tau)d\tau
        = x(t) * h(t)$$
    \subsubsection*{Kausalität}
    Bei einem Kausalen System ist $h(t) = 0$ für $t<0$

    \subsection*{Verzerrung}
    Verzerrung = Verformung des Eingangssignals (gefiltert, moduliert, gedämpft, entzerrt).
    Sie ist bei LTI Systemen \textbf{linear}.
    Eigenschaften:
    \begin{itemize}
        \item Amplitudenverzerrung:
              \begin{itemize}
                  \item $|H(\omega)|$ unkonstant
                  \item Spektralanteile $X(\omega)$ in $Y(\omega)$ nicht gleich gross
                  \item $Y(\omega)$ hat keine zusätzlichen Spektralanteile
              \end{itemize}
        \item Phasenverzerrung:
              \begin{itemize}
                  \item $\varphi(\omega)$ nicht linear zu $\omega$
                  \item Spektralanteile nach Frequenz zeitlich ungleich verzögert
                  \item Signal im Zeitbereich $x(t) \neq y(t)$
              \end{itemize}
    \end{itemize}
    \textbf{Verzerrungsfrei:} Ausgangssignal wird nur um einen Faktor $k_a$ skaliert
    oder um eine Zeit $t_d$ verzögert. somit gilt:
    \begin{itemize}
        \item Übertragungsfunktion: $H(\omega) = k_a \cdot e^{-j\omega z_d}$
        \item   Amplitudengang: $|H(\omega)| = k_a$
        \item  Phasengang: $\varphi_h(\omega) = -t_d \cdot \omega$
    \end{itemize}

    \subsubsection{Frequenzantwort}
    Die \textbf{Frequenzantwort} $\bm{H(\omega)}$ ist die Fouriertransformierte Impulsantwort.
    Sie ist eine komplexwertige dimensionslose Gewichstsfunktion.
    Auch sie beschreibt ein LTI-System vollständig.

    $$ Y(\omega) = X(\omega) \cdot H(\omega)$$

    \subsubsection*{Berechnung des Ausgangssignals}
\begin{enumerate}
   \item Fourier-Transformation:  $X(\omega) = \mathcal{F}[x(t)]$ 
   \item Berechnung in Frequenz  $Y(\omega) = X(\omega) \cdot H(\omega)$ 
   \item Rücktransformation:  $y(t) = \mathcal{F}^{-1}[Y(\omega)]$ 
\end{enumerate}


    \subsubsection*{Bezeichnungen}
    Übertragungsfunktion: $H(\omega)=|H(\omega)| \cdot e^{j\varphi_H(\omega)}$ \\
    Amplitudengang: $|H(\omega)|$ \\
    Phasengang: $\varphi_H(\omega)$ \\

    \subsubsection*{Filtereigenschaften}
    $Y(\omega) = X(\omega) \cdot H(\omega)$ \\
    $|Y(\omega)| = |X(\omega)| \cdot |H(\omega)|$ \\
    $\varphi_y(\omega) = \varphi_x(\omega) + \varphi_H(\varphi)$

    \subsubsection*{Filtertypen}
    \begin{itemize}
        \item Tiefpassfilter: Alle Frequenzen \textbf{grösser} als Eckfrequenz $\omega_c$ werden weggefiltert.
        \\ $H(\omega) = 0$ für $|\omega| > \omega_c$
        \\ Bandbreite: Von Frequenz 0 Hz bis Eckfrequenz.
        \item Hochpassfilter: Alle Frequenzen \textbf{kleiner} als Eckfrequenz $\omega_c$ werden weggefiltert.
        \\ $H(\omega) = 0$ für $|\omega| < \omega_c$
        \item Bandpassfilter: Alle Frequenzen \textbf{ausserhalb} der Eckfrequenzen $\omega_{c1,2}$ werden weggefiltert.
        \\ $H(\omega) = 0$ für $|\omega| < \omega_{c1}$  und $H(\omega) = 0$ für $|\omega| > \omega_{c2}$
        \\ Bandbreite: Von unterer bis zur oberen Eckfrequenz
        \item Bandsperre: Alle Frequenzen \textbf{innerhalb} der Eckfrequenzen $\omega_{c1,2}$ werden weggefiltert.
        \\ $H(\omega) = 0$ für $\omega_{c1} < |\omega| < \omega_{c2}$
        \item Quadraturfilter: Schiebt die Phase um $-90^{\circ}$.
        $H(\omega) = -j \cdot sign(\omega)$
    \end{itemize}

\end{multicols}
