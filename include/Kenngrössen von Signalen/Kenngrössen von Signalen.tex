\section{Kenngrössen von Signalen}
\begin{tabular}{p{6cm}p{12cm}}
  Energie                                                            &
  $W_n = \lim_{T \to \infty}  \int \limits _{-T/2} ^{T/2} |f(t)|^2 dt$            \\
  \rowcolor{TabularBackgroundColor}
  Leistung                                                           &
  $P_n = lim_{T \to \infty} \frac{1}{T} \int \limits _{-T/2} ^{T/2} |f(t)|^2 dt$  \\
  *Linearer Mittelwert
  \newline \tiny(auch: $ \bar{x}, x_m$)                              &
  $X_0 = \frac{1}{T} \int \limits _{-T/2}^{T/2} x(t) dt $                         \\
  \rowcolor{TabularBackgroundColor}
  *Quadratischer Mittelwert                                          &
  $X^2 = \frac{1}{T} \int \limits _{-T/2}^{T/2} x^2(t) dt = {X_0}^2 + \sigma^2 $                        \\
  *Effektivwert \newline \tiny{("Quadratischer Mittelwert", RMS)}    &
  $X^2 = \frac{1}{T} \int \limits _{-T/2}^{T/2} \sqrt{x^2(t)} dt $                \\
  \rowcolor{TabularBackgroundColor}
  Mittelwert n. Ordnung \newline \tiny(nur Signale $\in \mathbb{R}$) &
  $X^n = \frac{1}{T} \int \limits _{-T/2} ^{T/2} x^n(t)dt$                        \\
  Varianz                                                            &
  $Var(x) = \sigma^2 = \frac{1}{T} \int \limits _{-T/2} ^{T/2} (x(t) - X_0)^2 dt$ \\
  \rowcolor{TabularBackgroundColor}
  Standardabweichung                                                 &
  $\sigma = \sqrt{Var(x)} = \sqrt{X^2 - (X_0)^2}$                                 \\
  \textbf{\tiny *Hinweis: Formeln sind für Klasse 2a angegeben. \newline
  Für Klasse 2b mit: $\lim_{t \to \infty}\;$ für Klasse 1: ohne $\frac{1}{T}$ }
\end{tabular}

\subsubsection*{Die Autokorrelationsfunktion (AKF)}
Für Klasse 2a
$$ \varphi_{xx}(\pm \tau) = \frac{1}{T} \int \limits _{-T/2} ^{T/2} x(t) \cdot x(t- \tau) dt
  = \frac{1}{T} \int \limits _{-T/2} ^{T/2} x(t + \tau) \cdot x(t)dt$$
\textbf{Eigenschaften:}
\begin{itemize}
  \item $\varphi_{xx}(0) = X^2 = (X_0)^2 + \sigma^2$ \tiny Quadratischer Mittelwert \normalsize
  \item $\varphi_{xx}(\tau) = \varphi_{xx}(\tau \pm n \cdot T)$, mit $n \in \mathbb{N}$,
        AKF hat gleiche Periode $T$ wie $x(t)$
  \item $\varphi_{xx}(\tau) = \varphi_{xx}(-\tau)$, AKF ist gerade Funktion
  \item $\varphi_{xx}(0) \geq \left|\varphi_{xx}(\tau)\right|$
  \item $\varphi_{xx}(\tau) \geq (X_0)^2 - \sigma^2$
  \item \textbf{Klasse 2b:} $\lim_{t \to \infty}$ voran; \textbf{Klasse 1:} $\lim_{t \to \infty}$ anstelle $\frac{1}{T}$
\end{itemize}

\subsubsection*{Die Kreuzkorrelationsfunktion (KKF)}
Für Klasse 2a:
$$ (x \star y)(\tau) = \varphi_{x_1x_2}(\tau) = \frac{1}{T} \int \limits _{-T/2} ^{T/2} x_1(t) \cdot x_2(t- \tau) dt
  = \frac{1}{T} \int \limits _{-T/2} ^{T/2} x_1(t + \tau) \cdot x_2(t)dt$$
\textbf{Eigenschaften:}
\begin{itemize}
  \item  $\varphi_{x_1x_2}(\tau) = \varphi_{x_1x_2}(-\tau)$
  \item ist \textbf{nicht} Kommutativ($(x \star y)(\tau) \neq (y \star x)(\tau)$)
  \item \textbf{Klasse 2b:} $\lim_{t \to \infty}$ voran; \textbf{Klasse 1:} $\lim_{t \to \infty}$ anstelle $\frac{1}{T}$
\end{itemize}

\subsection{Signalleistung, physikalische Leistung und Rauschen}
\begin{minipage}{0.45\textwidth}


  \begin{flalign*}
    k\textrm{[dB]} & = 10 \cdot \log_{10} (\frac{P_y}{P_x})
    \\ & = 10 \cdot \log_{10}(\frac{(y_{rms})^2}{(x_{rms})^2})
    \\ & = 20 \cdot log_{10}(\frac{y_{rms}}{x_{rms}})
  \end{flalign*}
  $$P_y = P_x \cdot 10^{\frac{k}{10}} \textrm{ und } P_x = \frac{P_y}{10^{\frac{k}{10}}}$$
  $$y_{rms} = x_{rms} \cdot 10^{\frac{k}{20}} \textrm{ und } x_{rms} = \frac{y_{rms}}{10^{\frac{k}{20}}}$$

\end{minipage}%
\begin{minipage}{0.45\textwidth}
  \begin{tabular}{l p{4cm}}

    Rauschleistung:           &
    $P_r = k \cdot T \cdot \Delta f $
    \\[5pt]
    \rowcolor{TabularBackgroundColor}
    Rauschspannung            &
    $U_r = \sqrt{4\cdot k \cdot T \cdot \Delta f \cdot R} $
    \\[5pt]
    Boltzmann-Konstante       &
    $k = 1.380662\cdot 10^{-23} \frac{J}{K}$
    \\[5pt]
    \rowcolor{TabularBackgroundColor}
    Signal-Rausch-Verhältnis: &
    $ a_r  = 10 \cdot log_10(\frac{P_s}{P_r})$
    \newline $= 20 \cdot log_10(\frac{U_s}{U_r})$
    \\[5pt]
    Rauschzahl:               &
    $F = \frac{P_{s_{in}}}{P_{r_{in}}}\cdot \frac{P_{r_{in}}}{P_{s_{out}}} $
    \\[5pt]
    \rowcolor{TabularBackgroundColor}
    logarithmisch:            &
    $A_F = 10 \cdot log_{10}(F)  $
    \newline $= a_{r_{in}} - a_{r_{out}}$
  \end{tabular}
\end{minipage}

\subsection{Amplitudenanalyse von Signalen}


\subsubsection{Amplitudendichte (WSK-Dichte)}

$$
  p(a)  = lim_{da \to 0} \frac{\sum t (a-\frac{da}{2} < x(t) \leq a + \frac{da}{2})}{T \cdot da}
  = \frac{1}{T} \cdot \frac{dt}{da}   
$$
\begin{minipage}{0.49 \textwidth}
  \subsubsection*{Mittelwerte} 
  \textbf{Linear:}
  
  \noindent $X_0 = \int \limits _{-\infty} ^{\infty} a \cdot p(a) da$
  \\
  \textbf{N-te Ordnung:}
  
  \noindent $X^n = \int \limits _{-\infty} ^{\infty} a^n \cdot p(a) da$
  
  \noindent \textbf{für Stochastische Signale}
  \\
  \noindent $p(a) = N(\mu, \sigma) = \frac{1}{\sigma \cdot \sqrt{2\pi}}\cdot e^{\frac{-(a-\mu)^2}{2\sigma^2}} $
  {\tiny Gauss verteilung} 
  \\
  wobei: $\mu$ = Linearer Mittelwert ($X_0$)
  \newline und $\sigma$ = Varianz
  
  \subsubsection*{Faltung zweier Amplitudendichten}
  
  $p(a) = \int \limits _{-\infty} ^{\infty} p_2(x) \cdot p_1(a-x)dx$
  \newline {\tiny Note: Faltung zweier Normalverteilungen = Normalverteilung}
\end{minipage}%
\begin{minipage}{0.49 \textwidth}
  
  
  \subsubsection*{Weitere Funktionen}
  
  \textbf{Q-Funktion}\\
  $Q(\xi) = \frac{1}{\sqrt{2\pi}}\int \limits _{\xi} ^{\infty} e^{-\frac{y^2}{2}}d\xi$
  \\ \textbf{Fehlerfunktion} \\
  $erf(\xi) = \frac{2}{\sqrt{\pi}} \int \limits _{0} ^{\xi} e^{-y^2} dy$
  \\ \textbf{komplementäre Fehlerfunktion} \\
  $erfc(\xi) =  \frac{2}{\sqrt{\pi}} \int \limits _{\xi} ^{\infty} e^{-y^2} dy $
\end{minipage}
  