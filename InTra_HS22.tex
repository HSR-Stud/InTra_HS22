% !TeX program = xelatex
% !TeX encoding = utf8
% !TeX root = InTra_HS22.tex

%% TODO: publish to CTAN
\documentclass[margin=normal]{tex/hsrzf}

%%%%%%%%%%%%%%%%%%%%%%%%%%%%%%%%%%%%%%%%%%%%%%%%%%%
% Packages

\usepackage{graphicx}
\usepackage{bm}
\usepackage{multicol}
\usepackage{pdfpages}
\usepackage{color, colortbl}
\usepackage{trfsigns}
\usepackage{mathrsfs}
\usepackage{adjustbox}
\usepackage{amsmath}

%% TODO: publish to CTAN
\usepackage{tex/hsrstud}

%% Language configuration
\usepackage{polyglossia}
\setdefaultlanguage[variant=swiss]{german}

%% License configuration
\usepackage[
    type={CC},
    modifier={by-nc-sa},
    version={4.0},
    lang={german},
]{doclicense}

% Color configuration

\definecolor{TabularBackgroundColor}{rgb}{0.83,0.96,0.96}
\definecolor{White}{rgb}{0.0,0.0,0.0}


%%%%%%%%%%%%%%%%%%%%%%%%%%%%%%%%%%%%%%%%%%%%%%%%%%%
% Metadata

\course{Elektrotechnik}
\module{InTra}
\semester{Herbstsemester 2022}

\authoremail{joel.Leirer@ost.ch}
\author{\textsl{Joël Leirer} -- \texttt{\theauthoremail}}

% did someone help you with this work?
\contributors{
  % I created this template, does that count?
  % do not forget to add yourself!
}

\title{\texttt{\themodule} Zusammenfassung}
\date{\thesemester}

%%%%%%%%%%%%%%%%%%%%%%%%%%%%%%%%%%%%%%%%%%%%%%%%%%%
% Document

\begin{document}

% use roman numberals for introductiory pages
\pagenumbering{roman}

\maketitle

% \begin{abstract}
% \end{abstract}

% show the names of the people who contributed to this document.
% \section*{Contributors}
% \thecontributors

\section*{Lizenz}
\doclicenseThis

\subsection*{Note:}
Erlaubte Hilfsmittel HS22:
\begin{itemize}
  \item 4 A4-Seiten Zusammenfassung
  \item Formelsammlung Fourier-/Laplacetransformation 
  \\ {\tiny(Diese Formelsammlung ohne Inhaltsverzeichnis, Anhang ist erwähnte Fourier-Tabelle.)}
\end{itemize} 


\newpage
\tableofcontents

% actual content
\clearpage
\setcounter{page}{1}
\pagenumbering{arabic}

%%%%%%%%%%%%%%%%%%%%%%%%%%%%%%%%%%%%%%%%%%%%%%
%Includes + Defines
%%%%%%%%%%%%%%%%%%%%%%%%%%%%%%%%%%%%%%%%%%%%%%

%\usepackage{color, colortbl}
%\usepackage{trfsigns}
%\usepackage{graphicx}
%\definecolor{TabularBackgroundColor}{rgb}{0.83,0.96,0.96}
%\usepackage{mathrsfs}


%%%%%%%%%%%%%%%%%%%%%%%%%%%%%%%%%%%%%%%%%%%%%%
%Content
%%%%%%%%%%%%%%%%%%%%%%%%%%%%%%%%%%%%%%%%%%%%%%
\section{Integraltransformationen}

\subsubsection*{Faltung}
\begin{multicols*}{2}
  
  $$y(t) = (x_1 * x_2)(t) = \int \limits _{-\infty} ^{\infty} x_1(\tau) \cdot x_2(t-\tau) d\tau$$
  \textbf{Eigenschaften:} \\
  \begin{tabular}{cc}
    Kommutativ & ($f * g = g * f$)\\
    Assoziatiov &($(f*(g*h)) = ((f*g)*h)$)\\
    Distributiv &($f*(g+h)= f*g + f*h$)\\
    Stetigkeit & ist \textbf{immer} stetig
  \end{tabular}
  
\subsubsection*{Vorgehen}
\begin{enumerate}
  \item $\tau$ bzw. $t-\tau$ in die Entsprechende Funktion Einsetzen
  \item Bereiche Bestimmen an denen die Funktion $\neq 0$ 
  \item Bereiche bzw. eingesetzte Funktionen in Hilfsdiagramm einzechnen (siehe Bsp.)
  \item Einzelne Integrationsbereiche mit Hilfe Diagramm bestimmen,
  indem Zeit $t$ "raufgezählt" und übergänge der Grenzen beachtet wird
  \item Integrale Bestimmen, Integralgrenzen = Eingezeichnete Grenzen im Diagramm.
\end{enumerate}
\includegraphics[width = 5cm]{include/Integraltransformationen/img/Faltungsgrenzen.png}
\textbf{Beispiel} 
\begin{enumerate}
  \item 
\end{enumerate}

\end{multicols*}

\subsection{Fourier-Reihe und Transformation}
\subsubsection*{Fourier-Reihe}
 \begin{tabular}{p{4.5cm}p{14.5cm}}
  Trigonometrische Form &
  $x(t) = \frac{a_0}{2} + \sum \limits _{n = 1} ^{\infty} a_n \cdot cos(2\pi n f_0 \cdot t) + b_n \cdot sin(2\pi n f_0 \cdot t) $
  \newline $a_n = \frac{2}{T} \int \limits _{T} x(t) dt$
  \newline $a_n = \frac{2}{T} \int \limits _{T} x(t) \cdot cos(2\pi n f_0 \cdot t)dt$
  \newline $b_n = \frac{2}{T} \int \limits _{T} x(t) \cdot sin(2\pi n f_0 \cdot t)dt$
  \\
  \rowcolor{TabularBackgroundColor}
  Harmonische Form      &
  $x(t) = r_0 + \sum \limits _{n = 1} ^{\infty} r_n \cdot cos(2\pi n f_0 \cdot t + \varpi_n)$
  \newline $r_0 = \frac{u_0}{2} = \frac{1}{T} \int  \limits _{T} x(t) dt $
  $r_n > 0 = \sqrt{u_n^2 + v_n^2}$
  $\varphi = arg(u_n - j \cdot v_n) $
  \\
  Komplexe Form         &
  $x(t) = \sum \limits _{n= -\infty} ^{\infty} c_n \cdot e^{jn2\pi f_0 \cdot t}$
  \newline $c_n =\overline{c_{-n}} = \frac{1}{T} \int \limits _{0} ^{T} x(t) \cdot e^{-j n 2 \pi f_0 \cdot t} dt$
  \newline $ 2\pi f_0$ wird auch als \textbf{Kreisfrequenz}  $\omega$ bezeichnet, \newline $f_0$ = Frequenz des Grundsignals $x(t)$. 
  \\
  \rowcolor{TabularBackgroundColor}
  Umrechnung Koeffizienten &
  $c_n =\overline{c_{-n}} = \frac{a_n - jb_n}{2} (n = 0,1,2,3,..., b_0 =0)$
  \newline $a_n = 2 \cdot Re(c_n); \; b_n = -2 \cdot Im(c_n) (n = 0,1,2,3,..., b_0 =0)$\\
\end{tabular}

\begin{multicols}{2}
  
  \subsubsection*{Fouriertransformation $\mathcal{F}(\omega)$}
  $$ X(\omega) = \mathcal{F}[x(t)] = \int \limits _{-\infty} ^{+\infty} x(t) \cdot e^{-j \omega t} dt $$
  $$ x(t) = \mathcal{F}^{-1}[X(\omega)] = \frac{1}{2 \pi} \int \limits _{- \infty} ^{+ \infty} X(\omega) \cdot e^{j \omega t} d\omega$$
  Rechenregeln Siehe Anhang.


  \subsubsection*{Spektraldarstellung}
  \includegraphics[width = 6cm]{include/Integraltransformationen/img/Spektrum.png}
\end{multicols}

\subsection{Laplace-Transformation}
$$F(s) = \mathscr{L} {f(t)} = \int \limits _{0} ^{\infty} f(t) \cdot e^{-st} dt \textrm{ mit } s \in \mathbb{C}$$
$$f(t) = \mathscr{L}^{-1} {f(t)} = \frac{1}{2\pi j} \int \limits _{p_0-j\infty} ^{p_0 + j\infty} F(s) \cdot e^{st} ds$$
Rechenregeln Siehe Anhang.
\subsubsection*{Lineare Differenzialgleichungen lösen:}

{\huge TODO: }

\subsubsection*{Zusammenhänge}
\includegraphics[width = 5cm]{include/Integraltransformationen/img/Zusammenhang_Laplace.png}
\subsection{Hilbert-Transformation}
Hilbert Transformation ist die Anwendung eines Quadraturfilters. \\
Definition im \textbf{Zeitbereich:}
$$\hat{x}(t) = x(t) * \frac{1}{\pi t} = \frac{1}{\pi} \int \limits _{-\infty} ^{\infty} \frac{x(\tau)}{t-\tau} d\tau$$
Im \textbf{Frequenzbereich}:
$$\hat{X}(\omega) = X(\omega) \cdot H(\omega) = -j \cdot sgn(\omega) \cdot X(\omega)$$

%%%%%%%%%%%%%%%%%%%%%%%%%%%%%%%%%%%%%%%%%%%%%%
%Includes + Defines
%%%%%%%%%%%%%%%%%%%%%%%%%%%%%%%%%%%%%%%%%%%%%%

%\usepackage{color, colortbl}
%\usepackage{trfsigns}
%\usepackage{graphicx}
%\definecolor{TabularBackgroundColor}{rgb}{0.83,0.96,0.96}


%%%%%%%%%%%%%%%%%%%%%%%%%%%%%%%%%%%%%%%%%%%%%%
%Content
%%%%%%%%%%%%%%%%%%%%%%%%%%%%%%%%%%%%%%%%%%%%%%

\section{Wichtige Funktionen}

\begin{multicols}{2}
  \small
  \subsubsection*{Sprungfunktion (Heaviside)}
  \includegraphics[width = 5cm]{include/Wichtige Funktionen/img/Sprungfunktion.png}
  \\ $H(t) = u(t) = \begin{cases}
      0 \textrm{ für }  t<0,                                                     \\
      [\frac{1}{2} \textrm{ für }  t = 0,] \textrm{ \tiny(machmal: 1 für $t=0$)} \\
      1 \textrm{ für }  t >0.
    \end{cases}   $
  \\
  \subsubsection*{Diracimpuls \tiny (auch Impuls-/Deltafunktion,-Distribution)}
  \includegraphics[width = 5cm]{include/Wichtige Funktionen/img/Impulsfunktion.png}
  \\ \footnotesize
  Unendlich kurzer, normierter Impuls mit unendlicher Amplitude.
  \\Eigenschaften:\\
  \resizebox{0.4\textwidth}{!}{%
    \begin{tabular}{ccc}
      \hline \rowcolor{TabularBackgroundColor}
      1.  & $\delta(-t) = \delta(t) $                                                                                            & gerade Funktion                        \\
      \hline
      2.  & $\delta(-t+t_0) = \delta(t-t_0)$                                                                                     & symmetrisch                            \\
      \hline \rowcolor{TabularBackgroundColor}
      3.  & $\delta(at)= \frac{1}{|a|}\delta(t)$                                                                                 & Skalierung                             \\
      \hline
      4.  & $\delta(\frac{t-t_0}{a}) = |a| \cdot \delta(t-t_0)$                                                                  & Skalierung und Verschiebung            \\
      \hline \rowcolor{TabularBackgroundColor}
      5.  & $\delta(t-t_0)f(t) = f(t_0)\delta(t-t_0)$                                                                            & Abtastung                              \\
      \hline
      6.  & $\int \limits _{-\infty} ^{\infty} \delta(t-t_0)f(t)dt = f(t_0)$                                                     & Siebungseigenschaft                    \\
      \hline \rowcolor{TabularBackgroundColor}
      7.  & $\int \limits _{-\infty} ^{\infty}  A\cdot \delta(t)dt = A$                                                          & Spezialfall Siebungseigenschaft        \\
      \hline
      8.  & $\delta(t-t_0) * f(t) = f(t-t_0)$                                                                                    & Faltung                                \\
      \hline \rowcolor{TabularBackgroundColor}
      9.  & $\delta(t-t_1) * \delta(t-t_2) = \delta(t-t_1-t_2)$                                                                  & Faltung                                \\
      \hline
      10. & $\delta(t) = \frac{du(t)}{dt}$                                                                                       & Ableitung Einheitssprung               \\
      \hline \rowcolor{TabularBackgroundColor}
      11. & $ \delta(t) = \lim _{\omega \to \infty} \frac{sin(\omega t)}{\pi t} $                                                & Definition                             \\
      \hline
      12. & $ \delta(t) = \lim _{\epsilon \to \infty} \frac{\epsilon}{\pi(t^2 + \epsilon^2)} $                                   & Definition                             \\
      \hline \rowcolor{TabularBackgroundColor}
      13. & $\delta(t) = \lim _{\epsilon \to 0} \frac{e^{-t^2/\epsilon}}{\sqrt{(\pi \epsilon)}} $                                & Definition                             \\
      \hline
      14. & $t^n \frac{d^n \delta(t)}{dt^n} = (-1)^n n! \delta(t)$                                                               & Ableitung                              \\
      \hline \rowcolor{TabularBackgroundColor}
      15. & $f(t) * \frac{d\delta(t-t_0)}{dt} = \frac{df(t-t_0)}{dt}$                                                            & Faltung mit Ableitung                  \\
      \hline
      16. & $\frac{d\delta(t)}{dt} = \frac{\delta(t)}{-t} = \lim _{\epsilon \to 0} \frac{-2\epsilon t}{\pi(t^2 + \epsilon^2)^2}$ & 1. Ableitung $\delta(t)$ = ungerade F. \\
      \hline
    \end{tabular}}
  \subsubsection*{Signumfunktion (Vorzeichenfunktion)}
  \includegraphics[width = 5cm]{include/Wichtige Funktionen/img/Signumfunktion.png}
  \footnotesize
  \\ $sgn(t) = \begin{cases}
      -1 \textrm{ für }  t<0,  \\
      0 \textrm{ für }  t = 0, \\
      1 \textrm{ für }  t >0.
    \end{cases}   $                                                          \\
  \subsubsection*{Rampenfunktion}
  \includegraphics[width=5cm]{include/Wichtige Funktionen/img/Rampenfunktion.png}
  \footnotesize
  \\ $r(t) = \begin{cases}
      0 \textrm{ für } t \leq 0, \\
      t \textrm{ für } t > 0.
    \end{cases}$                                                          \\
  \subsubsection*{Rechteckimpuls}
  \includegraphics[width=5cm]{include/Wichtige Funktionen/img/Rechteckimpuls.png}
  \footnotesize
  \\ $p_a(t) = u(t+a)-u(t-a)= \begin{cases}
      1 \textrm{ für } |t| < a,           \\
      \frac{1}{2} \textrm{ für } |t| = a, \\
      0 \textrm{ für } |t| > a.
    \end{cases} $                               \\
  \subsubsection*{Dreieckimpuls}
  \includegraphics[width=5cm]{include/Wichtige Funktionen/img/Dreieckimpuls.png}
  \footnotesize
  \\ $\Lambda(t) = \begin{cases}
      1 - \frac{|t|}{a} \textrm{ für } |t| < a \\
      0 \textrm{ für } |t| \geq a
    \end{cases}$                                      \\
  \subsubsection*{Sinc-Funktion}
  \includegraphics[width=5cm]{include/Wichtige Funktionen/img/SincFunktion.png}
  \footnotesize
  \\ $sinc(t) = \frac{sin(t)}{t} \forall t$                                                                                   \\
\end{multicols}


%Needs Package: 
%\usepackage{bm}
%\usepackage{multicol}
\section{LTI-Systeme}
\begin{minipage}{0.5\textwidth}

    \subsection*{Linearität und Zeitinvarianz}
    \begin{itemize}
        \item $\mathcal{T}[x_1(t) + x_2(t)] = y_1(t) + y_2(t)$
        \item $\mathcal{T}[k_a \cdot x(t)] = k_a \cdot y(t)$
        \item $\mathcal{T}[x(t-t_0) = y(t-t_0)]$
    \end{itemize}

    \subsection{Beschreibung von LTI Systemen}

    \subsubsection{Impulsantwort}
    Systemreaktion auf $\delta(t)$.

    $ y(t) = \mathcal{T}[x(t)]
        = \int \limits _{-\infty} ^{\infty} x(\tau) \cdot h(t-\tau)d\tau
        = x(t) * h(t)$

    \subsubsection{Frequenzantwort}
    Fouriertransformierte Impulsantwort.
    \\ Auch Übertragungsfunktion genannt.

    $$ Y(\omega) = X(\omega) \cdot H(\omega) = X(\omega) \cdot G(j\omega)$$

    \subsubsection{Berechnung des Ausgangssignals}
    1. Integraltransformation:
    \newline $X(\omega) = \mathcal{F}[x(t)] {\; \big / \;}
        X(s) = \mathscr{L}[x(t)]$ \\
    2. Berechnung in Bild / Frequenz:
    \newline $Y(\omega) = X(\omega) \cdot H(\omega)  {\; \big / \;}
        Y(s) = X(s) \cdot G(s)$ \\
    3. Rücktransformation:
    \newline $y(t) = \mathcal{F}^{-1}[Y(\omega)] = \mathscr{L}^{-1}[Y(s)]$

\end{minipage}%
\begin{minipage}{0.6\textwidth}


    \subsection{Bezeichnungen}
    Übertragungsfunktion: $H(\omega) = G(j\omega) =|H(\omega)| \cdot e^{j\varphi_H(\omega)}$ \\
    {\tiny auch: Frequenzgang}\\
    Amplitudengang: $|H(\omega)|$ \\
    Phasengang: $\varphi_H(\omega)$ \\

    \subsubsection{Filtereigenschaften}
    $Y(\omega) = X(\omega) \cdot H(\omega)$ \\
    $|Y(\omega)| = |X(\omega)| \cdot |H(\omega)|$ \\
    $\varphi_y(\omega) = \varphi_x(\omega) + \varphi_H(\varphi)$

    \subsection{BIBO-Stabilität \tiny {Bound-Input-Bound-Output}}

    Systemantwort Begrenzt, wenn Eingangssignal Begrenzt
    \newline $\Rightarrow$ Konvergenzhalbebene Übertragungsfunktion enthält Imaginär-Achse
    \newline $\Rightarrow$ Alle Polstellen Übertragungsfunktion links von "$j$-Achse"
    \begin{center}
        \includegraphics[width = 4cm]{include/Integraltransformationen/img/BiBo.png}
    \end{center}

\end{minipage}%

%%%%%%%%%%%%%%%%%%%%%%%%%%%%%%%%%%%%%%%%%%%%%%
%Includes + Defines
%%%%%%%%%%%%%%%%%%%%%%%%%%%%%%%%%%%%%%%%%%%%%%

%\usepackage{color, colortbl}
%\usepackage{trfsigns}
%\usepackage{graphicx}
%\definecolor{TabularBackgroundColor}{rgb}{0.83,0.96,0.96}


%%%%%%%%%%%%%%%%%%%%%%%%%%%%%%%%%%%%%%%%%%%%%%
%Content
%%%%%%%%%%%%%%%%%%%%%%%%%%%%%%%%%%%%%%%%%%%%%%


\section{Integrieren und Differenzieren}
\subsection{Integrationsregeln}
\begin{tabular}{ll}
    Linearit\"at & $\int{f(\alpha x+\beta )dx=\frac{1}{\alpha}\cdot F(\alpha x+\beta)+C}$ \\
    Partielle Integration & $\int\limits_a^b{u'(x)\cdot v(x)dx}=\biggl[
    u(x)\cdot v(x) \biggr]_a^b-\int\limits_a^b{u(x)\cdot v'(x)dx}$ 
    \tiny($v(x)$ = einfacheste Funktion wählen!) \normalsize\\
    
    Substitution (Rationalisierung) & $t=\tan\frac{x}{2}, \qquad
    dx=\frac{2dt}{1+t^2} \qquad \sin  x=\frac{2t}{1+t^2} \qquad \cos x=\frac{1-t^2}{1+t^2}
    \quad\int{R(\sin(x)\cos(x))dx}$\\

    Allgemeine Substitution &
    $\int\limits_{a}^{b}{f(x)dx}=\int\limits_{g(a)}^{g(b)}{f(g(t))\cdot
    g'(t)dt}\qquad x=g(t)\qquad g'(t)=\frac{dt}{dx}\qquad dx=\frac{1}{g'(t)}\cdot dt$\\
    
    Logarithmische Integration & $\int{\frac{f'(x)}{f(x)}dx}=\ln|f(x)|+C 
    \qquad{(f(x)\neq 1)}$\\

    Spezielle Form des Integranden & $\int{f'(x)\cdot
    (f(x))^{\alpha} dx}= f(x)^{\alpha +1}\cdot \frac{1}{\alpha+1}+C
    \qquad{(\alpha \neq -1)}$\\

    Differentiation & $\int \limits ^{b} _{a} {f'(t)dt}=f(b)-f(a)$\qquad
    $\frac{d}{dx} \int \limits ^{x} _{1} {f(t)dt}=f(x)$
  \end{tabular}

\subsection{Ableitungsregeln}
\begin{tabular}{ll}
Linerität & $(\lambda f + \mu g)'(x) = \lambda f'(x) + \mu g'(x) \; \forall \lambda, \mu \in \mathbb{R} $ \\
Produktregel & $(f\cdot g)' = f'g + fg'$\\
Quotientenregel & $(\frac{f}{g})' = \frac{f'g - fg'}{g^2}$\\
Kettenregel & $ (f(g))' = f'(g) \cdot g' $\\
Potez & $((x-a)^n)'= n\cdot(x-a)^{n-1}$\\
Trigo & $sin'''' = cos''' = -sin'' = -cos' = sin$\\
\end{tabular}


\subsection{Partialbruchzerlegung}
Nenner Faktorisieren, bei mehreren gleichen Termen steigt exponent des Nenners
Zähler = Polynom mit einem Grad kleiner als Nenner
Zähler Gleichsetzen, Mit Gleichungssystem nach A,B,C.. auflösen.
Ansätze:
$$\frac{\dots}{x(x-3)^2} = \frac{A}{x} + \frac{B}{x-3} + \frac{C}{(x-3)^2}$$
$$\frac{\dots}{(x-2)^3} = \frac{A}{x-2} + \frac{B}{(x-2)^2} + \frac{C}{(x-2)^3}$$
$$\frac{\dots}{x^4 + x^2} = \frac{A}{x} + \frac{B}{x^2} + \frac{Cx + D}{x^2 + 1}$$

\section*{Wichtige Werte \& Vereinfachungen}

\subsubsection*{Integration über Periodendauer}
$$\int_T (x \cdot sin(\omega t + \alpha))^2 dt = x^2 \cdot \int_T sin^2(\omega t +\alpha) dt = \frac{x^2}{2}$$
$$\int_T (x \cdot cos(\omega t+\alpha))^2 dt = x^2 \cdot \int_T cos^2(\omega t+\alpha) dt = \frac{x^2}{2}$$


\subsubsection*{Komplex sin/cos}

$$cos \varphi = \frac{e^{j\varphi}+ e^{-j\varphi}}{2}, \; sin \varphi = \frac{e^{j\varphi} - e^{-j\varphi}}{2j} $$




\newpage
\includepdf[pages=-]{AnhangPDF/fourierLaplaceTabelle_v02.pdf}

\end{document}
